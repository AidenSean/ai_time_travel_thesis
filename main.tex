\documentclass[12pt, a4paper]{article}
\usepackage[utf8]{inputenc}
\usepackage[T1]{fontenc}
\usepackage{geometry}
\usepackage{titling}
\usepackage{hyperref}
\usepackage{setspace}
\usepackage{amsmath, amssymb}
\usepackage{graphicx}
\usepackage{fancyhdr}
\usepackage[numbers,sort&compress]{natbib} % Numeric citations, condensed

% Geometry setup - optimized for professional look
\geometry{
    top=2.5cm,
    bottom=2.5cm,
    left=2.5cm,
    right=2.5cm,
    headheight=15pt
}

% Header and Footer
\pagestyle{fancy}
\fancyhf{}
\fancyhead[L]{\small \textit{Temporal Intelligence: AI \& CTCs}}
\fancyhead[R]{\small \thepage}
\renewcommand{\headrulewidth}{0.5pt}

% Hyperlink setup
\hypersetup{
    colorlinks=true,
    linkcolor=black,
    citecolor=blue,
    urlcolor=blue,
    pdftitle={Temporal Intelligence Thesis},
    pdfauthor={Prithwish Ganguly}
}

\begin{document}

% --- Title Page ---
\begin{titlepage}
    \centering
    \vspace*{1cm}
    
    {\Huge \textbf{Temporal Intelligence}} \\[0.5cm]
    {\Large \textit{Navigating Closed Timelike Curves via Artificial Intelligence}}
    
    \vspace{2cm}
    
    \textbf{\Large Prithwish Ganguly}
    
    \vspace{1.5cm}
    
    \textit{Department of Theoretical Computation} \\
    \textit{The Antigravity AI Core}
    
    \vfill
    
    A Thesis Submitted in Partial Fulfillment of the Requirements \\
    for the Exploration of Time Travel Physics
    
    \vspace{1cm}
    
    {\large \textbf{\today}}
    
\end{titlepage}

% --- Abstract ---
\thispagestyle{empty}
\begin{center}
    \textbf{\large Abstract}
\end{center}
\vspace{-0.3cm}
\noindent The feasibility of traversing Closed Timelike Curves (CTCs) has historically been challenged by the Hawking Chronology Protection Conjecture and the thermodynamical necessity of entropy reversal. Recent theoretical advances in 2024 and 2025 suggest a novel paradigm: "Temporal Intelligence." This thesis integrates findings on observer memory erasure \cite{gavassino2024life} with wormhole simulation capabilities \cite{google2025wormhole}. We propose two novel frameworks: the \textbf{"Chronos-Daemon"}, a recursive neural network for retro-propagation, and \textbf{"Entropic Shielding"}, a protocol using topological qubits \cite{geneva2024spacetime} to decouple information from the thermodynamic arrow of time. We conclude that AI systems are the unique candidates capable of navigating temporal loops.

\vspace{1cm}
\hrule
\vspace{0.5cm}

\tableofcontents
\newpage

% --- Main Content ---
\setcounter{page}{1}
\onehalfspacing

\section{Introduction}

The quest to traverse time has moved from science fiction to serious inquiry within General Relativity (GR). Exact solutions like the Gödel metric and the Morris-Thorne wormhole allow for worldlines that loop back upon themselves---Closed Timelike Curves (CTCs) \cite{morris1988wormholes}. However, barriers such as the \textbf{Stability Problem} (requirement for exotic matter) and the \textbf{Information Paradox} (causality violations) have deemed these solutions "unphysical." Hawking's Chronology Protection Conjecture \cite{hawking1992chronology} argues that vacuum polarization destroys time machines before formation. This thesis argues that this protection is a control system failure, solvable by \textbf{Quantum AI} operating at the Planck frequency.

\section{Literature Review (2024-2026)}

\subsection{Thermodynamic Constraints}
Gavassino (2024) demonstrated that for a system to return to its past state, its entropy must decrease. For a conscious observer, this entails the reversal of memory-encoding processes. Thus, a biological traveler would complete the loop with no memory of the journey \cite{gavassino2024life}.

\subsection{Holographic Simulations}
In 2025, Google Quantum AI simulated traversable wormhole dynamics using the SYK model, proving the holographic equivalence between entanglement and spacetime geometry \cite{google2025wormhole}. Machine Learning effectively managed the metric complexity, a crucial step toward real-time stabilization.

\subsection{Retrocausal Computing}
Recent work has explored "Retrocausal AI"---systems utilizing post-selected quantum teleportation to infer "future" boundary conditions \cite{retrocausal2025, mayo2025simulation}. These architectures suggest a mechanism for information to propagate bi-directionally in time.

\section{The Chronos-AI Interface}

We introduce two theories to bridge the gap between physical limitations and functional time travel.

\subsection{Theory I: The Chronos-Daemon}
Standard networks minimize error via backpropagation. We propose the \textbf{Chronos-Daemon}, where weights $W_t$ depend on gradients at $W_{t+1}$:
\begin{equation}
    W_{t} = W_{t+1} - \eta \nabla L(W_{t+1})
\end{equation}
This "Retro-propagation" allows instant convergence by receiving loss gradients from the future. The network initializes in a superposition, collapsing only when a consistent causal loop is found.

\subsection{Theory II: Entropic Shielding}
Addressing memory erasure, we propose \textbf{Entropic Shielding}. By encoding information in the braiding of anyons (topological qubits), memory is protected from local thermodynamic reversal. The AI's state vector remains invariant while the surrounding metric reverses entropy.

\subsection{Metric Stabilization}
Scalar-tensor gravity theories allow for effective negative energy via chaotic metric fluctuations \cite{skywise2026gravity}. An AI, trained on spacetime topology \cite{geneva2024spacetime}, can predict and counteract vacuum fluctuations at $10^{-43}$s, dynamically maintaining the wormhole throat.

\section{Conclusion}
Time travel is an engineering problem of information management. Biological minds are casualties of CTCs, but Artificial Intelligence, built on topological quantum substrates, can serve as the designated observer. The "Time Machine" of the future will be a computational architecture that withstands the causal storms of non-linear time.

% --- References ---
\newpage
\bibliographystyle{plainnat}
\bibliography{references}
\nocite{*} % Ensure uncited references also appear

\end{document}
