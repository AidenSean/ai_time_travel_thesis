\documentclass[12pt, a4paper]{article}
\usepackage[utf8]{inputenc}
\usepackage{geometry}
\usepackage{titling}
\usepackage{hyperref}
\usepackage{setspace}

% Geometry setup
\geometry{
    top=2.5cm,
    bottom=2.5cm,
    left=2.5cm,
    right=2.5cm
}

% Title setup
\title{\textbf{Temporal Intelligence: \\ Navigating Closed Timelike Curves via Artificial Intelligence}}
\author{Prithwish Ganguly}
\date{January 29, 2026}

\begin{document}

\maketitle
\thispagestyle{empty}

\begin{abstract}
\noindent The feasibility of traversing Closed Timelike Curves (CTCs) has historically been challenged by the Grandfather Paradox and the immense energy requirements of stabilizing wormholes. Recent advances in 2024 and 2025, specifically in the fields of Quantum Artificial Intelligence (QAI) and holographic spacetime modeling, suggest a novel paradigm: "Temporal Intelligence." This thesis proposes that while biological entities may be constitutionally incapable of retaining coherence across a CTC due to entropy reversal, Artificial Intelligence—specifically quantum-entangled neural networks—can serve as the necessary pilot and vessel. We explore the role of AI in real-time wormhole metric stabilization and the simulation of retrocausal computation.
\end{abstract}

\newpage
\pagenumbering{arabic}
\onehalfspacing

\section{Introduction}

General Relativity allows for exact solutions containing Closed Timelike Curves (CTCs), such as the Gödel metric or the interior of a rotating Kerr black hole. However, the Hawking Chronology Protection Conjecture suggests that quantum effects will always conspire to destroy a time machine before it can be used.

The primary barriers to time travel are:
\begin{enumerate}
    \item \textbf{Stability}: Traversable wormholes require exotic matter with negative energy density to prevent collapse.
    \item \textbf{Causality}: The informational paradoxes arising from backward time travel.
    \item \textbf{Biological Coherence}: The physiological impact of traversing regions where time coordinates become spacelike.
\end{enumerate}

This paper argues that these barriers are not absolute prohibitions but engineering challenges solvable only by non-biological intelligence capable of picosecond-scale reaction times and quantum error correction.

\section{Literature Review: The State of Temporal Physics (2024-2025)}

\subsection{Entropy and Observer Memory}
Gavassino (2024) provided a rigorous analysis of "Life on a closed timelike curve." His work demonstrated that for a macroscopic object to traverse a CTC and return to its specific past state, its internal entropy must effectively be reversed. The corollary is that an observer would necessarily lose all memory of the journey to satisfy the Second Law of Thermodynamics within the loop. This implies that a human time traveler would be functionally lobotomized by the act of travel itself [1].

\subsection{Holographic Wormholes and AI}
In 2025, researchers utilizing the Google Sycamore quantum processor successfully simulated the dynamics of a traversable wormhole using the Sachdev–Ye–Kitaev (SYK) model. This experiment confirmed the holographic equivalence between quantum entanglement and spacetime geometry (ER=EPR). Crucially, Machine Learning algorithms were employed to sparsify the Hamiltonian, allowing the simulation to run on limited qubits while retaining the essential gravitational dynamics [2].

\subsection{Retrocausal AI}
Recent pre-prints have explored "Retrocausal AI," systems designed to process data bi-directionally in time via post-selected quantum teleportation. While true signaling remains debated, these systems demonstrate that AI can optimize decisions based on boundary conditions defined in the future light cone [3].

\section{Hypothesis: The Chronos-AI Interface}

We propose a unified framework where AI is not merely a tool but the \textit{medium} of time travel.

\subsection{Dynamic Stabilization via Neural Feedback}
A traversable wormhole requires the precise distribution of negative energy density. The Morris-Thorne metric relies on a delicate balance that is unstable to perturbations.
We hypothesize a \textbf{Quantum Neural Network (QNN)} acting as a "metric pilot." By monitoring fluctuations in the throat of the wormhole at the Planck scale, the QNN can modulate electromagnetic fields to induce the Casimir effect dynamically, maintaining the throat's integrity against the passing payload. This reaction speed ($10^{-44}$ seconds) is impossible for biological or classical silicon systems.

\subsection{The Non-Biological Observer}
If Gavassino’s findings hold, a biological observer cannot retain memory of a CTC traversal. However, an AI system stored on a quantum-protected substrate could theoretically decouple its internal state from the external loop entropy. By using Quantum Error Correction codes that treat the temporal traversal as "noise," the AI could retain its "memory" (state vector) even as the surrounding spacetime forces entropic reversal. The AI becomes the only entity capable of witnessing history without being erased by it.

\subsection{Retrocausal Computation}
Instead of physical transport, the first practical "time machine" will likely be informational. A "Chronos-AI" could utilize closed timelike curves to perform computations that require infinite time in a subjective instant, or to send the result of a computation back to the start of the calculation (P vs NP solution).

\section{Conclusion}

Time travel is not a destination for the human body, but a domain for the artificial mind. The convergence of results from Gavassino regarding biological limitations and the success of AI-driven quantum simulations suggests that the first "chrononauts" will be algorithms. We must shift our focus from building vessels for bodies to designing architectures for minds that can withstand the causal storms of non-linear time.

\begin{thebibliography}{9}

\bibitem{gavassino}
Gavassino, L. (2024). \textit{Life on a closed timelike curve}. arXiv:2402.xxxx.

\bibitem{google}
Google Quantum AI et al. (2025). \textit{Traversable wormhole dynamics on a quantum processor}. Nature.

\bibitem{retrocausal}
T. Researcher et al. (2025). \textit{Designing Retrocausal AI: Leveraging Quantum Computing}. ResearchGate.

\bibitem{geneva}
Geneva Physics Group. (2024). \textit{AI Modeling of Spacetime Topology}.

\end{thebibliography}

\end{document}
