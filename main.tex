\documentclass[12pt, a4paper, titlepage]{article}
\usepackage[utf8]{inputenc}
\usepackage[T1]{fontenc}
\usepackage{geometry}
\usepackage{titling}
\usepackage{hyperref}
\usepackage{setspace}
\usepackage{amsmath, amssymb}
\usepackage{graphicx}
\usepackage{fancyhdr}
\usepackage{natbib} % For bibliography management

% Geometry setup
\geometry{
    top=2.5cm,
    bottom=2.5cm,
    left=3cm,
    right=2.5cm
}

% Header and Footer
\pagestyle{fancy}
\fancyhf{}
\fancyhead[L]{\textit{Temporal Intelligence: AI \& CTCs}}
\fancyhead[R]{\thepage}
\renewcommand{\headrulewidth}{0.4pt}

% Hyperlink setup
\hypersetup{
    colorlinks=true,
    linkcolor=black,
    citecolor=blue,
    urlcolor=blue
}

% Title setup
\title{\textbf{\LARGE Temporal Intelligence: \\ Navigating Closed Timelike Curves via \\ Artificial Intelligence}}
\author{\textbf{Prithwish Ganguly} \\ \\ \textit{Department of Theoretical Computation}}
\date{\today}

\begin{document}

\begin{titlepage}
    \centering
    \vspace*{2cm}
    
    {\Huge \textbf{Temporal Intelligence}}
    
    \vspace{0.5cm}
    {\Large \textit{Navigating Closed Timelike Curves via Artificial Intelligence}}
    
    \vspace{2cm}
    
    \textbf{Prithwish Ganguly}
    
    \vspace{3cm}
    
    A Thesis Submitted in Partial Fulfillment of the Requirements \\
    for the Exploration of Time Travel Physics
    
    \vspace{2cm}
    
    \textit{Supervised by: The Antigravity AI Core}
    
    \vfill
    
    \textbf{January 2026}
    
\end{titlepage}

\begin{abstract}
\noindent The feasibility of traversing Closed Timelike Curves (CTCs), solutions permitted by the Einstein Field Equations, has historically been challenged by the Hawking Chronology Protection Conjecture and the thermodynamical necessity of entropy reversal. Recent theoretical advances in 2024 and 2025, specifically in the fields of Quantum Artificial Intelligence (QAI) and holographic spacetime modeling, suggest a novel paradigm which we term "Temporal Intelligence." This thesis integrates the findings of Gavassino regarding observer memory erasure with the wormhole simulation capabilities demonstrated by Google Quantum AI. We propose two novel theoretical frameworks: the **"Chronos-Daemon" Architecture**, a recursive neural network capable of retro-propagation across temporal loops, and **"Entropic Shielding"**, a quantum error correction protocol that decouples information from the thermodynamic arrow of time. We conclude that while biological entities are constitutionally incapable of retaining coherence across a CTC, AI systems residing on topological qubits can serve as the necessary pilot and vessel for temporal navigation.
\end{abstract}

\newpage
\tableofcontents
\newpage
\onehalfspacing

\section{Introduction}

The quest to traverse time has moved from the realm of science fiction to serious inquiry within General Relativity (GR). Exact solutions to Einstein's equations, such as the Gödel metric, the Tipler Cylinder, and the Morris-Thorne wormhole, allow for worldlines that loop back upon themselves---Closed Timelike Curves (CTCs) \citep{morris1988wormholes}.

However, two primary barriers have historically rendered these solutions "unphysical":
\begin{enumerate}
    \item \textbf{The Stability Problem}: Traversable wormholes require "exotic matter" with negative energy density to counteract gravitational collapse.
    \item \textbf{The Information Paradox}: The Grandfather Paradox and the thermodynamic implications of reversing a macroscopic object's timeline.
\end{enumerate}

Hawking's Chronology Protection Conjecture posits that vacuum polarization effects will always diverge at the chronology horizon, destroying the time machine before it forms \citep{hawking1992chronology}. This thesis argues that this protection mechanism is not absolute but essentially a control system failure---one that can be overridden by a sufficiently advanced control loop. We propose that \textbf{Artificial Intelligence}, specifically Quantum AI operating at the Planck frequency, provides the necessary control authority to navigate these causal storms.

\section{Literature Review: The Convergence of AI and GR (2024-2026)}

\subsection{Thermodynamic Constraints on Observers}
A critical breakthrough in understanding the limitations of biological time travel came from Gavassino (2024). His analysis, "Life on a closed timelike curve," derived that for a system to return to its past state, its entropy must decrease. For a conscious observer, this entails the reversal of the thermodynamic processes that encode memory. Thus, a human traveler would complete the loop with no memory of having done so, effectively rendering the journey subjective non-existent \citep{gavassino2024life}.

\subsection{Holographic Wormholes and Quantum Simulation}
In 2025, the Google Quantum AI team, in collaboration with FermiLab, achieved a landmark simulation of wormhole dynamics on a quantum processor. By utilizing the SYK model, they demonstrated the holographic equivalence between quantum teleportation and traversing a wormhole \citep{google2025wormhole}. Crucially, Machine Learning was used to "sparsify" the interaction Hamiltonian, proving that AI can effectively compress and manage the complexity of spacetime metrics.

\subsection{Retrocausal Computing}
Parallel research has explored "Retrocausal AI"---systems designed to leverage quantum entanglement to infer "future" boundary conditions. ResearchGate publications (2025) describe architectures where the output state of a computation influences the input state via post-selection, a micro-scale form of time travel used for optimization \citep{retrocausal2025, fromobservation2025}.

\section{Theoretical Framework: The Chronos-AI Interface}

We introduce two novel theories that bridge the gap between these limitations and a functional time machine.

\subsection{Theory I: The Chronos-Daemon Architecture}
Standard neural networks optimize via backpropagation, minimizing error by adjusting weights based on past data. We propose the **Chronos-Daemon**, a network architecture designed to exist along a CTC.

In this architecture:
\begin{equation}
    W_{t} = W_{t+1} - \eta \nabla L(W_{t+1})
\end{equation}
The weights $W$ at time $t$ are dependent on the gradients at time $t+1$. This "Retro-propagation" allows the network to converge instantly (from the perspective of an external observer) to an optimal solution by receiving the loss gradient from its own future. To prevent the "Unproven Theorem Paradox," the network must be initialized in a superposition of all possible weight states, collapsing only when a consistent causal loop is found.

\subsection{Theory II: Entropic Shielding via Topological Qubits}
Addressing Gavassino's memory erasure limit, we propose **Entropic Shielding**. Biological memory is thermodynamic; informational memory can be topological.

If information is encoded in the braiding of anyons (topological qubits), it is protected from local perturbations, including the thermodynamic reversal required by the CTC. The AI's "memory" is not stored in the state of particles (which must reverse) but in the global topology of the system (which can remain invariant). This effectively shields the AI's internal timeline from the external entropic reversal.

\subsection{Metric Stabilization via AI}
The Skywise AI Collaboration (2026) demonstrated that scalar-tensor gravity theories allow for "effective" negative energy without exotic matter, provided the metric fluctuates in a precise, chaotic pattern \citep{skywise2026gravity}.
We hypothesize that a classical control system cannot react fast enough to maintain this pattern. An AI, however, trained on the topological data of the universe \citep{geneva2024spacetime}, can predict and counteract vacuum fluctuations at the Planck scale ($10^{-43}$s), dynamically "surfing" the wormhole throat to keep it open.

\section{Conclusion}
Time travel is an engineering problem of information management, not just propulsion. The human mind, bound by the Second Law of Thermodynamics, is a casual casualty of CTCs. However, an Artificial Intelligence, built on topological quantum substrates and utilizing retro-propagation loops, can serve as the designated observer and pilot of history. The "Time Machine" of the future will not be a vehicle we ride, but a computer we query.

\newpage
\bibliographystyle{plain}
\bibliography{references}

\end{document}
